En conclusion, ce projet a permis de poursuivre le développement d'une machine visant à produire des écrans
de phase sur-mesure sur le lieu d'utilisation de ces derniers. Il a aussi servi à développer un tout nouveau
système de mesure et de caractérisation des écrans produit directement intégré dans le montage optique de test du laboratoire.

La nouvelle machine est bien plus accessible et compréhensible dans son utilisation grâce à son interface WEB utilisable sans fil et
agnostique de toute plateforme. Elle reste tout de même capricieuse dans les cas où elle se bouche et cela reste bloquant dans le processus
de fabrication d'écran. En effet, cela prend du temps de déboucher la machine et de nettoyer l'écran raté pour le refaire, de plus il y a cette
incertitude de savoir si la machine va se boucher après un certain temps de projection continue ou pas.

Le nouveau système de mesure accompagné de son logiciel est aussi une partie essentielle du projet, sans la possibilité d'effectuer des mesures automatiquement
il aurait été impossible d'obtenir des résultats probants. L'état actuel de l'appareil de mesure est satisfaisant, quelques fonctions pourraient être
adaptées et ajoutées, mais cela est faisable juste en modifiant le code du microprocesseur de ce dernier.
Concernant le logiciel de mesure, son apparence est satisfaisante et l'interface est suffisamment compréhensible, par contre niveau réactivité de l'interface on pourrait mieux faire, Matlab est très lourd dans certains
de ses processus (tel que le plot de graphiques). De plus comme l'API d'interaction avec la caméra n'est pas fait officiellement par Thorlabs, ils ne peuvent pas vraiment apporter
du support en cas de demandes spécifiques. Il s'agira dans le futur de rajouter des fonctions de confort et de traduire le programme sous python.

En conclusion, la majorité des objectifs de ce projet ont été remplis. Une machine de fabrication d'écran améliorée fut produite et mise en service et a pu produire des échantillons prometteurs, il ne manque plus que
le temps supplémentaire pour se familiariser avec cette dernière et faire des tests étendus. N'oublions pas l'appareil de mesures qui théoriquement pourrait mesurer l'entièreté de la surface d'un disque de phase, accompagné de son
logiciel de mesure à l'interace utilisateur permettant d'observer les coefficients de Zernike de façon dynamique et pouvant réaliser des mesures de façon autonome sans que l'utilisateur ne doive se soucier de quoi que ce soit.
Une fois la machine de production d'écran de phase mise correctement sur pieds, elle sera très utile pour les futurs développements d'optique adaptative de l'institut d'optique de l'école, sans se limiter à cela, ce design pourrait être répliqué de façon ouverte
ou commerciale à petite échelle, la majorité des éléments utilisés dans la machine de projection sont obtenables facilement et les design customs sont conçus pour être imprimés en 3D.
\vfil
\hspace{8cm}\makeatletter\@author\makeatother\par

\hspace{8cm}\begin{minipage}{5cm}
    \printsignature
\end{minipage}
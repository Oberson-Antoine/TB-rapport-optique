\section{Accomplissements}
Ce projet a permis d'accomplir les objectifs suivants :
\begin{itemize}
    \item Mettre en fonction la boite de fabrication des écrans d'origine.
    \item Trouver une solution d'amélioration de la machine.
    \item Réalisation d'un système de motorisation de la rotation et de la translation des écrans.
    \item 2 PCBs fonctionnels servant au bon fonctionnement des deux système produits.
    \item Création d'une UI pour la fabrication des écrans.
    \item Création d'un logiciel de mesure des écrans avec affichage dynamique des données.
    \item Création d'un script matlab d'analyse des données.
\end{itemize}
Concernant l'objectif de réalisation de deux écrans de bonne qualité, il n'a pas pu être atteint, en effet l'analyse
des données extraites des écrans produits n'est pas satisfaisante.

\section{Améliorations}
Plusieurs améliorations sont souhaitables à l'avenir.
\subsection{Système de projection}
Lors de la fabrication des écrans, la buse a eu tendance à se boucher, une des solutions appliquées
pour réduire ce problème est de remplacer une des parts d'eau du mélange 4:1 d'acrylique par une part d'alcool, isopropylique,
la même chose a été faite pour la solution de nettoyage en y ajoutant une part d'alcool. Cela permet de liquéfier les grumeaux présents dans
le liquide, mais cela n'est pas parfait, plusieurs pistes s'offrent à nous :
\begin{itemize}
    \item Changer le liquide à projeter par soit un liquide pour aérographe, soit pour la version non protectrice aux UV de vernis de chez Lascaux.
    \item Essayer de diluer le vernis dans de l'acétone, car c'est aussi un des solvants possibles pour le vernis Lascaux.
    \item Commander chez Spray Systems une aiguille dite de "Shut-off/Cleanout needles" permettant de contrôler le flux de liquide mais aussi d'éliminer les bouchons.
\end{itemize}

D'autres améliorations de confort sont possibles :
\begin{itemize}
    \item Ajout d'un ventilateur d'extraction de fumée avec un filtre à charbon actif.
    \item Ajout d'un laser "horizontal" pour mieux visualiser le nuage de fumée.
    \item Ajouter d'un système d'éclairage dans le caisson.
\end{itemize}

\subsection{Système de mesure}
Plusieurs améliorations au système de mesure seraient pertinentes à faire :
\begin{itemize}
    \item Refaire le programme de mesure sous python (découverte de cette possibilité tard dans le projet).
    \item Ajouter une fonction de prise de mesure de référence automatique.
    \item Améliorer le script d'analyse des données pour le rendre plus confortable à utiliser (par exemple avoir une interface, pouvoir comparer plusieurs séries de mesures,...).
\end{itemize}
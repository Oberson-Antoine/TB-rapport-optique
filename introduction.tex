%L'introduction est une section requise dans un rapport technique. Introduisez votre travail, l'idée de départ et les objectifs attendus. Un lecteur qui découvrirait votre projet au travers de cette introduction devrait ainsi être capable d'en comprendre le cadre, l'idée générale et les aboutissants du projet.

En 2014, le laboratoire d'optique (Optolab.iai) de l'HEIG-VD a produit le design optique et les instruments de 1ère génération du \href{https://atasam.atauni.edu.tr/}{DAG (Dogu Anadolu Gözlemevi, Eastern Anatolia Observatory)}\footnotemark
un télescope optique et infrarouge proche de 4m de diamètre situé en Turquie à 3000m d'altitude dans les montagnes Palandöken proche de la ville d'Erzurum en Anatolie de l'Est. Plus précisément aux collines de Karakaya, cette région fut sélectionnée pour sa faible humidité, le vent y est faible et sa direction est stable,
le nombre de jours et nuits clairs, une ville est à proximité mais, elle n'est pas visible depuis la localisation du DAG, il y a donc peu de pollution lumineuse, ces caractéristiques atmosphériques et structurelles en font donc un endroit propice à l'installation de plusieus grands télescopes permettant entre autres l'observation
dans l'infrarouge proche (ce qui est possible dans peu d'endroits au monde).

Malheureusement, même avec les conditions atmosphériques et géographiques les plus parfaites, le télescope fait toujours face à un grand ennemi : \textbf{les turbulences}.
Le travail actuel d'Optolab est donc de développer un système d'optique adaptative ayant pour but de contrer les turbulences optiques produites par l'atmosphère. Ce système
d'optique adaptative a besoin d'être testé à l'aide d'écrans circulaires permettant de simuler des turbulences optiques.

Dans le cadre de ce projet de Bachelor, il convient donc de reprendre le travail réalisé dans un projet de Master précédent de l'HES-SO, ce dernier consistait en une boîte permettant de projeter un gaz d'acrylique transparente sur une plaque de plastique, elle aussi, transparente. La turbulence du nuage de gaz
d'acrylique immortalise sur la plaque en plastique une différence de chemin optique similaire à la distribution de la turbulence optique.

Le but de ce travail de Bachelor est donc de comprendre le fonctionnement de la machine de projection pour en suite l'améliorer afin de produire in-situ des écrans de phase calibrés.
\footnotetext{\url{https://atasam.atauni.edu.tr/}}

\newpage


\section{Contexte}
Cette section \underline{n'est pas obligatoire}, mais elle est souvent présente dans un rapport technique pour compléter l'introduction et définir le contexte du travail \cad le cadre formel dans lequel le travail est mené.



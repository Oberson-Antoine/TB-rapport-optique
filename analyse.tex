Que peut-on retirer de ces séries de mesures ?

\section{Écran 15 secondes}
Premièrement, parlons des séries consécutives de l'écran sprayé pendant 15 secondes.
La série 1 semble sortir de l'ordinaire, en effet cette dernière n'a pas la même forme générale et ne semble pas avoir
la même pente que les deux autres. Le paramètre de Fried $r_{0_1}$ (0.006937 cm) de cette série est quant à lui 10 fois plus petit que
ceux des mesures 2 et 3 ($r_{0_2} = 0.03245 cm$ et $r_{0_3} = 0.01015 cm$).

Comme ces séries de mesures ont été faites de manière consécutive sur le même diamètre, on doit s'attendre à obtenir des résultats similaires,
comme ceux des séries 2 et 3. La série 1 est donc une anomalie. Dû à soit une contamination externe : passage dans la salle, lumières aux alentours qui s'allument,
défaut logiciel de mesure(?).

Est-ce que ces écrans de phases sont utilisables et correspondent à nos attentes ?

Malheureusement non, les courbes de variance sont trop différentes dans leur forme par rapport à la variance de Noll.
Cela s'observe sur la courbe "fit" produite à partir du $r_0$ calculé, seule une partie du graphique des variances des mesures semble
correspondre puis une divergence s'observe, pas la même pente, pas les mêmes "marches d'escaliers".
\label{sec:invalidation_écrans}
Toutefois cela reste de bonne augure, on peut tout de même deviner qu'on est sur la bonne piste.
Mais que la structure d'écran évoquée dans la \autoref{sec:model_ecran_invalide}, sous forme de petites gouttes d'acrylique transparentes espacées entre
elles est mauvaise. Une structure en "tartine" serait plus souhaitable, avec une couche entière mais non-homogène d'acrylique sur les écrans.

\section{Écrans 30 et 60 secondes}
Comparons ces deux écrans, on voit que plus un écran est exposé, plus son paramètre de fried sera petit, à \textbf{30 secondes} $r_{0_{30}} = 0.001949 cm$ et à \textbf{60 secondes}
$r_{0_{60}} = 0.001509 cm$. Il est aussi possible d'observer que les piques correspondants à peu près à chaque palier de la courbe des variances de Noll s'applatissent de plus en plus en fonction de l'exposition.
Une autre observation prometteuse est le fait que la pente de la courbe des variances des mesures se rapproche de celle de la courbe des variances de Noll si l'exposition au spray est plus longue.

Cela confirme le fait que la structure d'écran de phase en petite gouttelettes d'acrylique espacée entre elles n'était pas la bonne solution.
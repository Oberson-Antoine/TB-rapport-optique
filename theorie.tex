\section{Fondamentaux}
Cette section parcour les notions fondamentales d'optique pour comprendre la suite de ce rapport.

\subsection{Les turbulences atmosphériques}

\subsubsection{Cause physique}
L'atmosphère de notre planète n'est pas un bloc homogène, cette dernière est composée de \textbf{masses d'air}. Ces masses d'air
sont des zones de l'atmosphère où les conditions de température, de pression et d'humidité sont homogènes \cite{masse_air_wiki}\cite{masse_air_unige}.
L'écoulement de ces zones se fait en régime turbulent à des vitesses usuellements mesurées entre 1 et 20 m/s sur des longueurs de 10 à 1000m, ce phénomène
turbulent des mouvements de masses d'air sera qualifié de \textbf{turbulance dynamique} dans la suite de ce rapport.

Cette turbulence dynamique (la vitesse de l'écoulement ) n'a pas d'incidence directe sur la propagation des ondes lumineuses, seul l'indice de réfraction de l'atmosphère influence la propagation
de la lumière. L'indice de réfraction est influencé par la densité de l'air selon la loi de Dale-Gladstone:

% \begin{equation}
%     N = 77.6 (\frac{P}{T}) + 3.74\cdot 10^5 (\frac{e}{T^2}) + C\frac{n_e}{f^2}
% \end{equation}

% Où:
% \begin{itemize}
%     \item $P$ = pression exprimée en hPa
%     \item $T$ = température absolue (K)
%     \item $e$ = pression de vapeur d'eau contenue dans l'air (hPa)
%     \item $C$ = $4,03 \cdot 10^{-7} m^{-3}Hz^2$
%     \item $n_e$ = densité électronique
%     \item $f$ = fréquence du signal
% \end{itemize}

\begin{equation}
  n-1 \sim \rho = \alpha_n \cdot \frac{P}{T}
\end{equation}

Où:
\begin{itemize}
  \item $P$ = pression exprimée en $N/m^2$
  \item $T$ = température absolue (K)
  \item $\alpha_a$ = $80\cdot10^{-8} K/Pa$
\end{itemize}

Pour aller plus loin, c'est généralement la température qui aura une grande influence sur l'indice de réfraction de l'air
et non la pression, car cette dernière est constante dans toute l'épaisseur d'une couche turbulente. Cette perturbation d'indice de
réfraction entre les couches turbulantes est donc la source des \textbf{Turbulences optiques}\cite{thèse_laurent_turbulence}.

\newpage
\subsubsection{Types de turbulences}
Plusieurs types de turbulences optiques sont observables à des altitudes bien spécifiques,
il est possible d'en distinguer 4 types :

\say{

  \begin{itemize}
    \item \textbf{1. Turbulence de coupole :}
          \newline
          Elle apparait lorsque l'air n'a pas la même température à l'interieur et à l'extérieur de la coupole (du téléscope). Cette dernière
          correspond à la turbulence du miroir.

    \item \textbf{2. Turbulence de surface :}
          \newline
          Active sur les premiers 10 à 100m. Elle doit son origine au refroidissement - par convection - du sol chauffé par le rayonnement solaire.
          Typiquement, elle atteint un minimum juste après le lever du Soleil, puis augmente régulièrement jusqu'au début de l'après-midi. Elle décroit
          ensuite pour atteindre un second minimum après le coucher du Soleil, puis augmente légèrement durant la nuit. Le moyen de minimiser au maximum cette turbulence
          est le choix du site d'installation du téléscope. Par exemple placer le téléscope en haut d'une tour loin de toute surface minimisera le phénomène.

    \item \textbf{3. Turbulence de moyenne altitude 1-5/6 Km :}
          \newline
          Elle trouve son origine à la fois dans les perturbations orographiques (onde générées par les reliefs) des courants atmosphériques et dans les instabilités thermiques
          de l'atmosphère. Elle est constituée par une multitude de fines couches turbulentes de quelques centaines de mètres d'épaisseur. Ici encore, seule la sélection du site d'accueil est susceptible
          de minimiser cette composante.

    \item \textbf{4. Turbulence de tropopause et stratosphérique :}
          \newline
          Au delà de $\approx 6 $Km, on observe une certaine systématique pour la plupart des sites :
          la turbulence atteint un minimum entre 6 et 10 Km, puis augmente à nouveau pour atteindre un maximum dans la tropopause (zone de transition entre la troposphère et la stratosphère), 10-20 Km. Cette couche
          turbulente est due à la présence de forts vents cisaillant l'atmosphère, très fréquents à la limite de la troposphère. La turbulence optique diminue en suite dans la stratosphère, pour finalement disparaître
          au-delà de 25 à 30 Km d'altitude.
  \end{itemize}

} \footnotemark


\footnotetext{\cite{thèse_laurent_turbulence_types}\fullcite{thèse_laurent_turbulence_types}}

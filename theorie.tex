\section{Fondamentaux}
Cette section parcour les notions fondamentales d'optique pour comprendre la suite de ce rapport.

\subsection{Les turbulances atmosphériques}

\subsubsection{Cause physique}
L'atmosphère de notre planète n'est pas un bloc homogène, cette dernière est composée de \textbf{masses d'air}. Ces masses d'air
sont des zones de l'atmosphère où les conditions de température, de pression et d'humidité sont homogènes \cite{masse_air_wiki}\cite{masse_air_unige}.
L'écoulement de ces zones se fait en régime turbulent à des vitesses usuellements mesurées entre 1 et 20 m/s sur des longueurs de 10 à 1000m, ce phénomène
turbulent des mouvements de masses d'air sera qualifié de \textbf{turbulance dynamique} dans la suite de ce rapport.

Cette turbulence dynamique (la vitesse de l'écoulement ) n'a pas d'incidence directe sur la propagation des ondes lumineuses, seul l'indice de réfraction de l'atmosphère influence la propagation
de la lumière. L'indice de réfraction est influencé par la densité de l'air selon la loi de Dale-Gladstone:

% \begin{equation}
%     N = 77.6 (\frac{P}{T}) + 3.74\cdot 10^5 (\frac{e}{T^2}) + C\frac{n_e}{f^2}
% \end{equation}

% Où:
% \begin{itemize}
%     \item $P$ = pression exprimée en hPa
%     \item $T$ = température absolue (K)
%     \item $e$ = pression de vapeur d'eau contenue dans l'air (hPa)
%     \item $C$ = $4,03 \cdot 10^{-7} m^{-3}Hz^2$
%     \item $n_e$ = densité électronique
%     \item $f$ = fréquence du signal
% \end{itemize}

\begin{equation}
  n-1 \sim \rho = \alpha_n \cdot \frac{P}{T}
\end{equation}

Où:
\begin{itemize}
  \item $P$ = pression exprimée en $N/m^2$
  \item $T$ = température absolue (K)
  \item $\alpha_a$ = $80\cdot10^{-8} K/Pa$
\end{itemize}

Pour aller plus loin, c'est généralement la température qui aura une grande influence sur l'indice de réfraction de l'air
et non la pression, car cette dernière est constante dans toute l'épaisseur d'une couche turbulente. Cette perturbation d'indice de
réfraction entre les couches turbulantes est donc la source des \textbf{Turbulences optiques}\cite{thèse_laurent_turbulence}.

